\documentclass{article}
\usepackage[pdftex]{graphicx}
\usepackage[utf8]{inputenc}
\usepackage{fancyhdr}
\usepackage{sfmath}
\usepackage{comment}
\usepackage[T1]{fontenc}
\usepackage[pdftex=true, pdftoolbar=true, pdfmenubar=true, pdfauthor = {}, pdfcreator = {PDFLaTeX}, pdftitle = {}, colorlinks=true, urlcolor=blue, linkcolor=blue, citecolor=blue, implicit=true, hypertexnames=false]{hyperref}
\usepackage[scaled]{helvet}
\renewcommand*\familydefault{\sfdefault}
\usepackage{booktabs}
\usepackage{tabu}
\usepackage{longtable}
\usepackage{siunitx}
\sisetup{
  round-mode          = places, % Rounds numbers
  round-precision     = 2, % to 2 places
  table-format        = 3
}
\usepackage[top=0.6in, bottom=0.6in, left=1in, right=1in, includeheadfoot]{geometry}
\usepackage{array}
\usepackage[english]{babel}
\newcolumntype{B}[2]{>{#1\hspace{0pt}\arraybackslash}b{#2}}
\setlength{\parindent}{0pt}
\usepackage[dvipsnames]{color}
\definecolor{gray}{gray}{0.85}
\pagestyle{fancy}
\renewcommand{\headrulewidth}{0pt}
\renewcommand{\footrulewidth}{0pt}

\newcolumntype{d}{S}
\newcolumntype{J}{S[table-format=1]}
\newcolumntype{K}{S[table-format=2]}
\newcolumntype{M}{S[table-format=4]}
\newcolumntype{N}{S[table-format=5]}
\newcolumntype{O}{S[table-format=6]}
% default is 3 so it's not included above
\usepackage{float}
\usepackage{marginnote}
\setlength\extrarowheight{2pt}
\newlength\mywidth
\setlength\mywidth{3.5in}
\usepackage{caption}
\captionsetup[table]{labelformat=empty}
\renewcommand*{\marginfont}{\scriptsize\itshape}
\fancyfoot{}
\fancyfoot[R]{\thepage}
\newcommand{\PreserveBackslash}[1]{\let\temp=\\#1\let\\=\temp}
\let\PBS=\PreserveBackslash
\newcommand{\longtablesep}{\endfirsthead \multicolumn{2}{c}{\textit{}} \\ \endhead \multicolumn{2}{c}{\textit{}} \\ \endfoot \endlastfoot}
\usepackage[titles]{tocloft}
\newcommand{\cftchapfont}{12}
\newcommand{\formatvardescription}[1]{#1}
\newcommand{\formatvarname}[1]{#1}
\newcommand{\formatvaralias}[1]{#1}
\newcommand{\formatvarfiltertext}[1]{\fontsize{8}{12}\textit{#1}}
\newcommand{\formatvarsubname}[1]{#1}
\usepackage{amsmath}
\usepackage{listings}
\usepackage{inconsolata}

\newenvironment{absolutelynopagebreak}
  {\par\nobreak\vfil\penalty0\vfilneg
   \vtop\bgroup}
  {\par\xdef\tpd{\the\prevdepth}\egroup
   \prevdepth=\tpd}

\fancyhead{}
\fancyhead[L]{{\fontsize{16}{24}\textbf{Edgar Anderson's Iris Data}}\fontsize{12}{18}\textbf{ \\ The data were collected by Anderson, Edgar}}


\begin{document}
\setlength{\tabcolsep}{1em}
\setlength{\LTleft}{0pt}
\setlength{\LTright}{\fill}
\setlength{\LTcapwidth}{\textwidth}
\vspace{.25in}

\begin{longtable}[l]{ll}
Sample  &  The irises of the Gaspe Peninsula \\ 
Conducted  &  1935 \\ 
\end{longtable}
This famous (Fisher's or Anderson's) iris data set gives the measurements in centimeters of the variables sepal length and width and petal length and width, respectively, for 50 flowers from each of 3 species of iris. The species are Iris setosa, versicolor, and virginica.


\renewcommand{\listtablename}{Table of Contents}
\listoftables
\clearpage

\begin{absolutelynopagebreak}
\begin{absolutelynopagebreak}
\textbf{Sepal Length}\hfill\textbf{\ttfamily{Sepal.Length}}

{\small numeric}

\vskip 0.10in
The length of the flower's sepal
\addcontentsline{lot}{table}{\parbox{1.8in}{\ttfamily{Sepal.Length}} Sepal Length}
\vskip 0.10in\end{absolutelynopagebreak} 
\begin{longtable}[l]{cccccc}
\toprule
{Mean} & {SD} & {Min} & {Max} & {n} & {Missing}\\
\midrule
5.84 & 1 & 4.3 & 7.9 & 150 & 0\\
\bottomrule
\end{longtable}\end{absolutelynopagebreak}
\begin{absolutelynopagebreak}
\begin{absolutelynopagebreak}

\vskip 0.25in
\textbf{Iris Species}\hfill\textbf{\ttfamily{Species}}

{\small factor}

\vskip 0.10in
The iris species
\addcontentsline{lot}{table}{\parbox{1.8in}{\ttfamily{Species}} Iris Species}
\vskip 0.10in\end{absolutelynopagebreak} 
\begin{longtable}[l]{JlK}
\toprule
{Code} & {Label} & {Count}\\
\midrule
1 & setosa & 50\\
2 & versicolor & 50\\
3 & virginica & 50\\
\bottomrule
\end{longtable}\end{absolutelynopagebreak}

\clearpage




\end{document}
